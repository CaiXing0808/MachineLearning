\documentclass{svproc}
%
% RECOMMENDED %%%%%%%%%%%%%%%%%%%%%%%%%%%%%%%%%%%%%%%%%%%%%%%%%%%
%

% to typeset URLs, URIs, and DOIs
\usepackage{url}
\def\UrlFont{\rmfamily}
\usepackage{graphicx,amsmath}
\usepackage{xcolor}
\usepackage{float}
\usepackage[section]{placeins}
\usepackage{subfigure}
\usepackage{caption}

\begin{document}
\mainmatter              % start of a contribution
%

\title{Report}
%
\titlerunning{ML_HW1, Yonglin Zhu(yzhu459)}  % abbreviated title (for running head)
%                                     also used for the TOC unless
%                                     \toctitle is used
%
\author{Yonglin Zhu\inst{1} }
%
%
%%%% list of authors for the TOC (use if author list has to be modified)

\institute{yzhu459, 902908165}
\maketitle              % typeset the title of the contribution
\section{Introduction}
Dataset1: Income
Problem 1 
Dataset2: Mushroom
Problem 2
\section{Data Preparatation}
%% Clean data 
%% test vs train
%%%%%%%%%%%%%%%%%%%%%%%%%%%%%%%%%%%%%%%%%%%%%%%%%%%%%%%%%%%%%%%%%%
\section{Method}
Your learning curve can be 
Train/test split rate
Depth of tree for Decision tree
number of features

\secsection{Random Forest}\label{sec1}

%%%%%%%%%%%%%%%%%%%%%%%%%%%%%%%%%%%%%%%%%%%%%%%%%%%%%%%%%%%%%%%%%%
\secsection{SVN}\label{sec1}
% use sklearn.svm.SVC to try more kernels
%%%%%%%%%%%%%%%%%%%%%%%%%%%%%%%%%%%%%%%%%%%%%%%%%%%%%%%%%%%%%%%%%%
\section{In Experiment 1, what is the expected value of our winnings after 1000 sequential bets? Explain your reasoning. Go here to learn about expected value: https://en.wikipedia.org/wiki/Expected\_value}\label{sec2}
According to the reason from above, the probability of winning $\$80$ is $1-1.76\times10^{-279}$. 
And the other case is losing $- \sum_{i=0}^{999} 2^i$ with the probability of $1.76\times10^{-279}$. 
Therefore the expected value of our winnings after 1000 sequential bets is $80\times(1-1.76\times10^{-279})+(- \sum_{i=0}^{999} 2^i)*(1.76\times10^{-279}) \approx 80$.

%%%%%%%%%%%%%%%%%%%%%%%%%%%%%%%%%%%%%%%%%%%%%%%%%%%%%%%%%%%%%%%%%%

\section{In Experiment 1, does the standard deviation reach a maximum value then converge or stabilize as the number of sequential bets increases? Explain why it does (or does
not).}\label{sec3}
Yes. The standard deviation reach a maximum value and the converage to 0. At the beginning of the simulation, the episode winnings in each simulation is very random. And then almost $100\%$ the winning will become $\$80$, which means the standard deviation will be 0, as shown in~\cite{fig1}.

\begin{figure}[ht!]
    \centering
            \centering
        \includegraphics[width=.98\textwidth]{../ex1SD.png}
        \label{fig1}
\end{figure}

\section{In Experiment 2, estimate the probability of winning $\$80$ within 1000 sequential bets. Explain your
reasoning.}\label{sec4}
In the experiment 2, I got 640 simulations with winning $\$80$ out of 1000 simulations. 
So my estimated the probability of winning $\$80$ within 1000 sequential bets is $64\%$. 
%%%%%%%%%%%%%%%%%%%%%%%%%%%%%%%%%%%%%%%%%%%%%%%%%%%%%%%%%%%%%%%%%%%


\section{In Experiment 2, what is the expected value of our winnings after 1000 sequential bets? Explain your reasoning.}\label{sec5}
From my experiment 2, I got 640 simulations with $\$80$ winning and 359 simulations with $\${-256}$ winning and one time with winning $\$42$. 
So that the expected value of our winnings after 1000 sequential bets can be $0.64\times80+0.359*({-256})+0.001*42 = {-40.662}$
%%%%%%%%%%%%%%%%%%%%%%%%%%%%%%%%%%%%%%%%%%%%%%%%%%%%%%%%%%%%%%%%%%%


\section{In Experiment 2, does the standard deviation reach a maximum value then converge or stabilize as the number of sequential bets increases? Explain why it does (or does not).}\label{sec6}
No. Unlike the case in experiment 1, the standard deviation has no obvious maximum. The standard deviation keeps increasing and converges to a value close to $\$160$.

\begin{figure}[ht!]
    \centering
            \centering
        \includegraphics[width=.98\textwidth]{../ex2SD.png}
                \label{fig2}
\end{figure}

\section{Include figures 1 through 5.}\label{sec7}
The figures 1 through 5 are attach as Fig.3 to 7 as below.
\begin{figure}[ht!]
    \centering
            \centering
        \includegraphics[width=.98\textwidth]{../ex1fig1.png}
                \label{fig3}
\end{figure}

\begin{figure}[ht!]
    \centering
            \centering
        \includegraphics[width=.98\textwidth]{../ex1fig2.png}
                \label{fig4}
\end{figure}

\begin{figure}[ht!]
    \centering
            \centering
        \includegraphics[width=.98\textwidth]{../ex1fig3.png}
                \label{fig5}
\end{figure}

\begin{figure}[ht!]
    \centering
            \centering
        \includegraphics[width=.98\textwidth]{../ex2fig4.png}
                \label{fig6}
\end{figure}

\begin{figure}[ht!]
    \centering
            \centering
        \includegraphics[width=.98\textwidth]{../ex2fig5.png}
                \label{fig7}
\end{figure}
\newpage
More info about this, go to: \url{http://quantsoftware.gatech.edu/Martingale}
%%%%%%%%%%%%%%%%%%%%%%%%%%%%%%%%%%%%%%%%%%%%%%%%%%%%%%%%%%%%%%%%%%%


\end{document}
